\documentclass[11pt, oneside]{article}   	% use "amsart" instead of "article" for AMSLaTeX format
\usepackage{geometry}                		% See geometry.pdf to learn the layout options. There are lots.
\geometry{letterpaper}                   		% ... or a4paper or a5paper or ... 
%\geometry{landscape}                		% Activate for rotated page geometry
%\usepackage[parfill]{parskip}    		% Activate to begin paragraphs with an empty line rather than an indent
\usepackage{graphicx}				% Use pdf, png, jpg, or eps§ with pdflatex; use eps in DVI mode
								% TeX will automatically convert eps --> pdf in pdflatex		
\usepackage{amssymb}
\usepackage{subeqnarray}
\newcommand{\bse}{\begin{subeqnarray}}
\newcommand{\ese}{\end{subeqnarray}}

%SetFonts

%SetFonts

\begin{document}

%\section{}
%\subsection{}

\centerline{{\bf Tutorial 5: Physics Informed Neural Networks (PINNs)} }
\vspace{1cm}

In this tutorial, you will learn to work with a PINN which is 
basically a form of neural differential equation. A common 
problem of neural networks (both FNNs and CNNs) is that 
the training data is rather limited. There  is a risk of overfitting 
to the small amount of data that one has and not being able 
to test the model accurately. As we have seen, a common 
approach  to avoid overfitting is regularisation but this 
may fail with limited data. PINNs  are a way to regularise a 
neural network in  a  more advanced way. Essentially, they 
help the neural network function have the right shape. To do this, 
one imbeds the network with information in the form of a 
differential equation. When there is little data available, being 
able to imbed additional information that isn’t data into the 
network is extremely powerful. In this tutorial, we will consider 
a PINN for a simple cooling problem (e.g. of water in a 
container), see notebook: Tutorial\_5.ipynb. 

%
\begin{itemize}
%
\item[(i)] 
Formulate the differential equation that is used in the notebook (in cell 2) 
for the cooling law? 
%
\item[(ii)]  
Next, two (classical) networks are trained with only N training data (10 
in the notebook), 
one with and one without regularisation.  Describe the differences between 
the results of these networks. Does this difference decrease with 
the amount of data points used for training? 
%
\item[(iii)]  What is the loss function that is used in the PINN? 
% 
\item[(iv)] Why is the performance of the PINN so much better
than the two classical  networks? How does this performance 
difference change with the number of data used for training? 
% 
\item[(v)]  Suppose now that the cooling rate, say $r$  is unknown 
in the equation. Describe the procedure on how $r$ can be 
determined from the data (as implemented in the notebook). 
%
\item[(vi)]  Study the accuracy of the cooling rate, as determined in 
(v),  versus the number of data points on which the network is 
trained. 

\end{itemize}

\end{document}  